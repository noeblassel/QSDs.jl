\documentclass{article}
\usepackage[utf8]{inputenc}
\usepackage{amsmath}
\usepackage{amssymb}
\usepackage{bbm}
\usepackage{nccmath}
\usepackage{caption}
\usepackage{subcaption}
\usepackage{graphicx}
\usepackage{hyperref}


%% custom commands
\newcommand{\cL}{\mathcal{L}}
\renewcommand{\d}{\mathrm{d}}
\newcommand{\e}{\mathrm{e}}
\newcommand{\E}{\mathbb{E}}
\renewcommand{\P}{\mathbb{P}}
\renewcommand{\div}{\mathrm{div}}
\newcommand{\R}{\mathbb{R}}
\newcommand{\n}{\mathrm{n}}

\newcommand{\uonetilde}{\widetilde{u}_{1,\beta}^{h_0,h_1}}
\newcommand{\uone}{u_{1,\beta}^{h_0,h_1}}
\newcommand{\utwo}{u_{2,\beta}^{h_0,h_1}}
\newcommand{\utwotilde}{\widetilde{u}_{2,\beta}^{h_0,h_1}}

\newcommand{\lambdaonetilde}{\widetilde{\lambda}_{1,\beta}^{h_0,h_1}}
\newcommand{\lambdaone}{\lambda_{1,\beta}^{h_0,h_1}}
\newcommand{\lambdatwo}{\lambda_{2,\beta}^{h_0,h_1}}
\newcommand{\lambdatwotilde}{\widetilde{\lambda}_{2,\beta}^{h_0,h_1}}

\newtheorem{conjecture}{Conjecture}

\title{Optimization of the parallel efficiency with respect to the domain}
\author{No\'e Blassel}
\begin{document}
\maketitle

    \section{The problem}
        We consider~$\Omega\subset \R^d$ an open bounded domain with smooth boundary,~$V\in \mathcal C^\infty (\overline \Omega)$, and~$\beta>0$.
        The interpretation of~$\Omega$ is that of a region of some larger configurational space, which is metastable for the dynamics
        \[\d X_t = -\nabla V(X_t)\,\d t + \sqrt{\frac2\beta}\, \d W_t,\]
        say, for simplicity, that~$V$ has a unique minimum~$x_*$ in~$\Omega$.
        Define its generator
        \[-\cL_\beta u = \nabla V \cdot \nabla - \frac1\beta \Delta,\]
        which we will consider to be operator acting on the weighted~$L^2$ space
        \[L^2_\mu(\Omega) = \left\{ v : \int_\Omega v^2 \e^{-\beta V} < \infty\right\},\]
        and endow it with Dirichlet boundary conditions:
        \[ \mathcal D(\cL_\beta) = H_{0,\mu}^1(\Omega)\cap H^2_\mu(\Omega).\]
        It is standard that~$\cL$ is positive, self-adjoint with compact resolvent, and thus its spectrum consists of real eigenvalues
        \[ 0 <  \lambda_1 < \lambda_2 \leq \dotsm\]
        tending to~$\infty$.

        The probability density of the QSD on~$\Omega$ is proportional to
        \[\e^{-\beta V}u_1,\,\text{in }\Omega,\]
        with~$u_1$ the solution to the eigenproblem
        \[-\cL_\beta u = \lambda u,\quad u\in H_{0,\mu}^1(\Omega),\]
        corresponding to the smallest eigenvalue~$\lambda_1$, which corresponds to the exit rate from~$\Omega$ starting from an initial condition distributed according to the QSD.

        The problem is that of identifying the domain~$\Omega$ for which the parallel replica algorithm is optimally efficient. 
        As the gain comes from the parallel exit step from the QSD, one wants to spend as much physical time in this step, or in other words one wishes to minimize the exit rate~$\lambda_1$. 
        On the other hand, if one enlarges the domain too much, metastability inside~$\Omega$ can become an issue, which will translate into a small relaxation rate~$\lambda_2 - \lambda_1$ to the QSD.
        Balancing these considerations motivate considering the objective function 
        \[\tilde J = \frac{\lambda_1}{\lambda_2-\lambda_1}.\]
        Equivalently, we minimize
        \[ J = \frac{\lambda_1}{\lambda_2}.\]
        Loosely, minimizing~$J$ with respect to~$\Omega$ yields a domain for which ParRep spends the smallest proportion of physical time in the decorrelation/dephasing step.
        Using shape optimization techniques, it can be shown that any locally optimal domain satisfies a relation on the normal derivatives of the first two eigenpairs of~$-\cL_\beta$:
        \[ \lambda_2\left(\frac{\partial u_1}{\partial n}\right)^2 = \lambda_1 \left(\frac{\partial u_2}{\partial n}\right)^2\qquad\text{on } \partial\Omega.\]

        In small dimensions, one can use this relation to implement gradient descent methods to optimize the domain, however this procedure is intractable for realistically-dimensioned systems.
        However, numerical experiments in small dimensions show that nearly optimal values of~$J$ are obtained by slightly fattenning the basin of attraction of~$x_*$ for the gradient dynamics
        \[ \dot X = -\nabla V(X).\]

        To understand this phenomenon, we consider a simpler model in one dimension.

    \section{A model problem in 1D}

    Consider a~$\mathcal C^\infty$ potential~$V:[a,b]\to \mathbb R$. Assume:

    \begin{enumerate}
        \item $V$ has exactly three critical points~$z_0 < x_* < z_1$ in~$(a,b)$.
        \item $V$ has a global minimum at~$x_*$, with~$\kappa_* = V''(x_*)>0$.
        \item $V$ has non-degenerate saddle points~$z_0,z_1$. We denote $$\kappa_0=-V''(z_0),\quad\kappa_1=-V''(z_1),$$ which are strictly positive.
        \item $V(x_*) < \min(V(a),V(b))$.
    \end{enumerate}

    We denote by~$\mu_\beta(\mathrm d x) = \mathrm e^{-\beta V(x)}\,\mathrm d x$ the (unnormalized) Gibbs measure, and consider the generator of the killed overdamped Langevin dynamics:
    Now let~$h_0 : \R_+ \to (0,z_0-a)$, $h_1:\R_+ \to (0,b-z_1)$, and consider the operator
    \[-\cL_\beta^{h_0,h_1} =  V' \partial_x - \beta^{-1}\partial_{xx}\]
    with domain $$\mathcal{D}(\cL_\beta^{h_0,h_1})=H_0^1(z_0-h_0(\beta),z_1+h_1(\beta)) \cap H^2(z_0-h_0(\beta),z_1+h_1(\beta)).$$ This operator is positive, self-adjoint with compact resolvent, and we denote
    $$ 0 < \lambda_{1,\beta}^{h_0,h_1} < \lambda_{2,\beta}^{h_0,h_1} \leq \dotsm $$
    its spectrum.

    \section{Conjectures}
    We list some conjectures based on numerical evidence about the asymptotic behavior of the first two eigenvalues and their associated eigenvectors.

    \begin{conjecture}\label{conj:lambda1}
        Let~$0<\delta<\frac12$,~$ z_0-a > h_0 > (\beta\kappa_0)^{\delta-\frac12}$,~$b-z_1>h_1>(\beta\kappa_1)^{\delta-\frac12}$.
         Then~$\lambda_{1,\beta}^{h_0,h_1},$ is asymptotically equivalent as~$\beta\to\infty$ to
        \begin{equation}
            \left\{
                \begin{aligned}
                    \frac{\sqrt{\kappa_0\kappa_*}}{2\pi}\e^{-\beta(V(z_0)-V(x_*))},&\quad V(z_0) < V(z_1),\\
                    \frac{\sqrt{\kappa_1\kappa_*}}{2\pi}\e^{-\beta(V(z_1)-V(x_*))},&\quad V(z_0) > V(z_1),\\
                    \frac{\sqrt{\kappa_0\kappa_*}+\sqrt{\kappa_1\kappa_*}}{2\pi}\e^{-\beta(V(z_0)-V(x_*))},& \quad V(z_0) = V(z_1).
                \end{aligned}
            \right.
        \end{equation}
    
    The content of this conjecture is that the exit time is doubled as soon as one extends the domain beyond the saddle points further than $O(\sqrt \beta)$. 
    We expect that this conjecture can be retrieved from the careful examination of the proofs of semiclassical estimates of~$\lambda_1$ (likely the condition on~$\delta$ will need to be strengthened).
    \end{conjecture}

    \begin{conjecture}\label{conj:lambda2}
        The limit
        \[\underset{\beta \to \infty}{\lim}\, \lambda_{2,\beta}^{0,0} = \lambda_{2,\infty} > 0\]
        is well-defined and furthermore,
        \[\underset{\beta \to \infty}{\lim}\, \lambda_{2,\beta}^{z_0-a,b-z_1}  = \lambda_{2,\infty}.\]

        An immediate corrolary is that for any~$0 < h_0 < z_0 -a$,~$0<h_1<b-z_1$,
        \[\underset{\beta \to \infty}{\lim}\, \lambda_{2,\beta}^{h_0,h_1}  = \lambda_{2,\infty},\]
        since eigenvalues are monotonous with respect to the domain by the Courant--Fischer principle (or by shape gradients computations).
    \end{conjecture}

    Proving these conjectures would provide an effective criterion to (approximately) maximize the parallel efficiency objective
    \[ J_{\beta}(h_0,h_1) = \frac{\lambda_{2,\beta}^{h_0,h_1}}{\lambda_{1,\beta}^{h_0,h_1}}\]
    in the vicinity of~$(0,0)$, since they would imply that~$J_\beta$ is asymptotically constant on $((\beta\kappa_0)^{\delta-\frac12},z_0-a)\times((\beta\kappa_1)^{\delta-\frac12},b-z_1)$, with value close to $2J_\beta(0,0)$.
    
    On the other hand, one should probably choose $h_0,h_1$ as small as possible so as to minimize the overlap between metastable states.
\end{document}